\chapter{Grundlagen}
\section{Django}
\label{Django}
Django ist ein Framework, mit dessen Hilfe Webseiten erstellt werden können. Dafür setzt Django auf die Programmiersprache Python und legt Wert auf Wiederverwendbarkeit der einzelnen Komponenten. Es folgt einem Model-Template-View Muster, welches sich an dem Model-View-Controller Schema orientiert. Mit Hilfe eines Objekt-relationalen Mappers (ORM) werden die geschriebenen Models in Datenbankstrukturen überführt und die Anfragen ausgeführt. Zum Testen der erstellten Webseite bringt Django einen Entwicklungs-Webserver mit. Django wird derzeit aktiv entwickelt, die neueste Version ist die 1.8 (Stand 06. April 2015).

\section{PostgreSQL}
Als Daten haltendes Backend haben wir das objektrelationale Datenbankmanagementsystem (DBMS) PostgreSQL \cite{PostgreSQL2015} ausgewählt. Das DBMS ist open-source und unterstützt einen Großteil des SQL-Standards. Zudem lässt sich PostgreSQL um eine Vielzahl von eigenentwickelten Komponenten erweitern.\\

\noindent Beispielsweise können folgende Komponenten selbst definiert werden:
\begin{itemize}
\item Datentypen
\item Funktionen
\item Operatoren 
\item u.v.m.
\end{itemize}