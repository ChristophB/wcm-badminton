\chapter{Erkenntnisse}
\label{Erkenntnisse}
Laut dem NTV-Artikel \cite{Hand2015}, wird der Anteil der Linkshänder in der Bevölkerung auf 10 bis 15\% geschätzt, ihr Anteil läge im Spitzensport in einigen Sportarten aber bei bis zu 55\%. Betrachtet man nun unsere mageren 8,81\% für den Anteil der Linkshänder, bekommt man den Eindruck, dass die Linkshänder im Nachteil wären. Eine kurze manuelle Auswertung der aktuellen Weltrangliste (Stand 18. Februar 2015) für das Herren- und das Dameneinzel für die ersten 25. Plätze, ergibt jedoch einen höheren Anteil der Linkshänder. Bei den Männern sind 6 der 25 Topleute Linkshänder, bei den Frauen sind es immerhin 4. Dies ergibt einen Anteil von 20\%, welcher immerhin etwas über dem Anteil an der Gesamtbevölkerung liegt und den vermeintlichen Nachteil in einen Vorteil wandelt. Dennoch sind die Daten mit Vorsicht zu genießen, da ebenfalls die Top-100 hätten ausgewertet werden können, was in der Folge wiederum zu anderen Zahl führt.

Für die Geschlechterverteilung ergibt sich ein geringes Ungleichgewicht zugunsten der Männer, die 59,8\% einnehmen. Bei der Länderverteilung belegt Indonesien mit 521 Spitzensportlern den ersten Platz, gefolgt von Großbritannien mit 363 Spielern. Deutschland belegt mit 199 Spielern den 9. Rang. Insgesamt lässt sich ein Trend in den asiatischen Ländern und in Europa erkennen, welche einen Großteil aller Spieler einnehmen.\\

Bei den Statistiken zu dem Verein, dem Coach, der gesprochenen Sprache und der Körpergröße lässt sich sagen, dass hier nur sehr wenige Daten vorhanden sind. Daher lässt sich keine allgemeine Aussage für diese Werte treffen.

\section{Fehler und Verbesserungsmöglichkeiten}
Der selbst geschriebene Crawler hält sich nicht an gängige Richtlinien, die ein "`richtiger"' Crawler einhalten würde. Beispielweise müsste man nach einer "`robots.txt"' suchen, damit man weiß welche Dokumente man crawlen darf und welche nicht. Außerdem müsste sich ein Crawler im HTTP-Protokoll über das Feld "`user-agent"' erkennbar machen, da es sich nicht um einen normalen Nutzer handelt. Um den Server nicht zu überlasten, müssten Wartezeiten eingehalten werden, bevor das nächste Dokument gecrawlt werden kann.\\

Im Bereich des Profil-Updates sind einige Verbesserungsmöglichkeiten denkbar. Man könnte beispielsweise ein Update für alle Werte erlauben, die sich im Laufe eines Lebens ändern können, auch wenn diese bereits gesetzt sind. Hierzu zählt auch der Nachname, der sich bei einer Heirat ändern kann. Für die Qualität der Datenbank wäre es wichtig, die geänderten Werte nicht einfach zu übernehmen, sondern als Vorschläge in einer gesonderten Tabelle abzuspeichern und manuell überprüfen zu lassen.\\

Verein, Coach als Text? Größe nur im Bereich 100-220 cm?

\section{Erweiterungsmöglichkeiten}
Es wäre denkbar die Webseite zu "`vervollständigen"', da in der Datenbank noch viele Informationen zu den Spielern leer sind. Dafür könnte man weitere Seiten crawlen und parsen, um die Lücken im Datenbestand zumindest teilweise zu schließen. Außerdem lassen sich weitere Informationen in die Webseite einpflegen. Beispielsweise könnten die Regeln für Badminton ergänzt oder interessante Matches zwischen den Top-Spielern als Video eingebettet werden. Selbstverständlich sind weitere Möglichkeiten für die Visualisierungen mit den vorhandenen Daten denkbar.